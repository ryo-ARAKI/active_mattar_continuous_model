\documentclass[12pt,dvipdfmx,svgnames,a4paper,uplatex]{ujarticle}
% +++++++++++++++++++++++++++++++++++++++++++
% パッケージの導入
% +++++++++++++++++++++++++++++++++++++++++++
%
% ===========================================
% 原稿設定
% ===========================================
% \usepackage{hyoshi}  % 表紙用スタイルファイル
% 原稿のサイズ
\usepackage[top=25truemm,bottom=25truemm,left=25truemm,right=25truemm]{geometry}
\usepackage{relsize}
%
% ===========================================
% 図・表関係
% ===========================================
\usepackage{graphicx}
\graphicspath{{./pics}}  % \includegraphicsで参照するディレクトリ
%
% ===========================================
% 参考文献
% ===========================================
\usepackage[url=false,isbn=false,doi=false,sorting=none,eprint=false]{biblatex}
\addbibresource{./ActiveMatter_ContinuousModel.bib}
%
% ===========================================
% 独自スタイルの導入
% ===========================================
\usepackage{/home/ryo/.config/LaTeX/mystyle}
{
  \theoremstyle{plain}
  \newtheorem{assumption}{仮定}
}
%
% ===========================================
% 表紙の記述
% ===========================================
\title{アクティブマター乱流のメゾスケール連続体モデル}
\author{荒木 亮}
\date{\today}

% +++++++++++++++++++++++++++++++++++++++++++
% 本文
% +++++++++++++++++++++++++++++++++++++++++++
\begin{document}
\maketitle
\tableofcontents

\section{連続体モデル方程式の導出}
\label{sec:derive_continuous_model}

本節では,先行研究~\cite{Wensink2012b}(とその supplemental material~\cite{Wensink2012a})および同じ著者らによる論文~\cite{Dunkel2013a}に基づいて,高密度なバクテリア懸濁液などのアクティブマターをよく記述する連続体モデルを導出する.


\subsection{仮定とその正当化}
\label{subsec:assumptions_justification}

連続体モデルを導出するために次を仮定する.

\begin{assumption}
  \label{as:incompressibility}
  十分に高密度なバクテリア懸濁液(や自己推進ロッド(SPR, \textbf{S}elf \textbf{P}ropelled \textbf{R}ods)の集合)は,非圧縮アクティブ流体としてよく近似できる.
\end{assumption}

\begin{assumption}
  \label{as:mean_velocity_field}
  バクテリア流体の本質的なダイナミクスは平均流速\(\vb*{v}(t,\vb*{x})\)で捉えられる~\footnote{ここでの「平均」は粗視化して連続場として捉えるという意味か?}.
\end{assumption}

仮定\ref{as:incompressibility}はバクテリア乱流の実験やSPRの数値計算において密度ゆらぎが小さいことから正当化できる.
仮定\ref{as:mean_velocity_field}には議論の余地がある.
アクティブマターはしばしば長距離相互作用をもち,このとき平均流速と平均の向きはdecoupleされているためである.
しかし,十分に高密度な環境下ではこれらはよくcoupleする.
そのため,仮定\ref{as:mean_velocity_field}は十分に高密度な環境で妥当としてよい.




\section{2次元安定性解析}


\section{方程式の無次元化}


\section{渦度方程式の導出}


\section{Fourierスペクトル法を用いた数値計算に向けて}


\printbibliography[title=参考文献]

\end{document}
